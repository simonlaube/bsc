% !TEX root = ../Thesis.tex
\chapter{Introduction}

%Test This is the introduction to the thesis template. The goal is to give students a starting point on how to format and style their Bachelor or Master thesis\footnote{This document also shows how to use the template.}. 

%Please make sure to always use the most current version of this template, by downloading it always from the original git repository:
%\begin{center}
	%\url{http://www.github.com/ivangiangreco/unibas-latex} 
%\end{center}

%We will use throughout this tutorial some references to Turing's imitation game~\cite{turing:1950} and the Turing machine~\cite{turing:1936}. You may be interested in reading these papers.

%\vspace{1em}
%The package comes with an option regarding the bibliography style.
%You can include the package with
%\begin{verbatim}
%\usepackage[citeauthor]{basilea}
%\end{verbatim}
%to be able to cite authors directly with
%\begin{verbatim}
%\citet{turing:1950}
%\end{verbatim}

% If the option is enabled, then the following reference should print Turing [2]:~\citet{turing:1950}

A network built of small, solar powered devices can be used for a wide variety of different applications. Some use-cases can be monitoring agricultural sites, recording weather data for scientific purposes or serve as an independent network for communication. For those devices (network nodes) to be autonomous, it is essential to reduce their energy consumption as much as possible. To transmit data between network nodes and to not having to connect the devices with cables, wireless technology with low power consumption is needed. In this project we use LoRa (Long Range) wireless technology which fulfils this requirement. Additionally to having a very small energy consumption, it is also capable of transmitting signals over distances greater than 15 kilometres under optimal conditions.~\cite{10.1007/978-3-030-01168-0_11} The tradeoff is a smaller data transmission rate. \\
To manage and propagate data through the network, we use a protocol called \textbf{TinySSB}. It is heavily influenced by the Secure Scuttlebutt Protocol which is an 'event-sharing protocol and architecture for social apps'. \cite{10.1145/3357150.3357396} TinySSB is optimized for small microcontroller devices by omitting data that is not essential while still guaranteeing authenticity and integrity (\cref{sec:anchor}). Both protocols use so-called \textbf{append-only-logs} or \textbf{feeds} to append new packets to. All data and metadata that is sent over the network is being stored in feeds (except for side hash chains, see \cref{sec:sidechain}). If a network node is able to verify or trust that a packet of a feed was created by the feed owner, it can autonomously verify if a new incoming packet is the correct continuation of the last received packet and therefore trust it as well. With this trust mechanism it is even possible to receive packets that have been propagated via a number of different nodes and guarantee that no middle man could have altered the data. However the drawback of this method is limited flexibility in changing or deleting old data. Since the devices that are used for this project have only limited storage capability we need to find ways to be able to remove old data while still keeping the secure properties of the append-only-logs. \\
In this thesis we will have a look at two different approaches of achieving this goal for two different use-cases. We will explore how we can use multiple different feeds and combine them to bigger constructs (feed-trees) using packets that serve as pointers to other feeds. These pointers help us to create a link between different feeds that can be differently interpreted depending on the type of the feed-tree. Those feed constructs use simple feeds and pointers as building blocks but can get quite complex when interacting with each other. We want to hide this complexity from the user and present him with only an abstraction of the feed-tree that can be used like a normal feed with additional functionality. \\
Furthermore we want to have a look at how we can improve resource management in general. This includes sorting incoming and outgoing packets according to priorities to reduce the number of packets sent over the network. Some feeds have to be given more computing time than others depending on how critical they are for either a specific feed construct or for the system as a whole. \\ While the focus of this project is not the real-life implementation of the device with solar cells, it should be as well prepared as possible for this scenario. Therefore the system needs to be able to withstand unexpected shutdowns due to power outages or other unforeseeable reasons and recover once it is started up again. \\



%We will explore how those two approaches can be implemented and how we can optimise them to limit the number of packets sent as much as possible. Additionally we want to have a look at how we can improve resource management by handling incoming and outgoing packets in a prioritised way.