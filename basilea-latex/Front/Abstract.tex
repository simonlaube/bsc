% !TEX root = ../Thesis.tex
\chapter{Abstract}
%In a wireless sensor network with intermittent connectivity, continuous operation requires unsupervised resource management. This project builds a demonstrator for sampling-rate adaption and data thinning that is dependent on memory availability and data drainage progress. The distributed system will consist of constrained devices running on solar power.

In a wireless sensor network consisting of devices running on solar power, connectivity is not guaranteed at all times. Depending on solar energy availability network nodes may or may not be able to communicate with each other. The software running on those devices therefore must be prepared to cope with unexpected shutdowns. In this project different methods are developed and implemented that can reduce energy consumption and limit storage usage. Thus resource management plays a central role and is especially important since the devices have to be run autonomously. TinySSB - a protocol that uses append-only-logs - is used as the basis of the software that is developed in this project.